\documentclass[a4paper,10pt]{article}
\usepackage[utf8]{inputenc}
\usepackage{amsmath}
\usepackage[margin=0.5in]{geometry}

\begin{document}

\section{Theory}

\subsection{Motion in the Trap}

We follow the treatment of Leibfried et al. where the potential of the ion trap is assumed to be harmonic:
\begin{equation}
 \Phi(x,y,z,t) = \frac{1}{2} U_{dc}(2z^2 - x^2 - y^2) + \frac{1}{2}U_{rf}(x^2 - y^2)\cos (\omega_{rf} t)
\end{equation}
In the expression above, $U_{dc}$ is the potential applied to the DC electrodes ($U$ in Leibfried) and $U_{rf}$ refers to the potential on the RF electrodes ($\tilde{U}$ in Leibfried), applied with radial frequency $\omega_{rf}$. We use the special choice of coefficients $\alpha=\beta=-1, \gamma=2$ and $\alpha'=-\beta'=1, \gamma'=0$.

The secular motion of the ion can be described with a pseudopotential approximation. The potential energy yields
\begin{equation}
 \Psi(x,y,z) =  \frac{1}{2}eU_{dc}(2z^2 - x^2 - y^2) + \frac{e^2U_{rf}^2}{4 m \omega_{rf}^2} (x^2 + y^2)
\end{equation}
From this, the trap frequencies are:
\begin{align}
 \omega_x &= \omega_y = \sqrt{\frac{e}{m}\left(\frac{e U_{rf}^2}{2m\omega_{rf}^2}-U_{dc} \right)}\\
 \omega_z &= \sqrt{\frac{2eU_{dc}}{m}}.
\end{align}

In this model, the radial trap frequencies are degenerate, which may lead to poor laser cooling. In order to lift the degeneracy of the radial modes, we
apply an additional static bias $U_{bias}$ on the RF electrodes as follows:
\begin{equation}
 \Phi(x,y,z,t) = \frac{1}{2} U_{dc}(2z^2 - x^2 - y^2) + \frac{1}{2}U_{bias}(x^2 - y^2) + \frac{1}{2}U_{rf}(x^2 - y^2)\cos (\omega_{rf} t)
\end{equation}
The Laplace's equation $\triangle\Phi= 0$ still holds with the additional bias term. The trap frequencies are now non-degenerate and are given by:
\begin{align}
 \omega_x &= \sqrt{\frac{e}{m}\left(\frac{e U_{rf}^2}{2m\omega_{rf}^2}-U_{dc} +U_{bias}\right)}\\
 \omega_y &= \sqrt{\frac{e}{m}\left(\frac{e U_{rf}^2}{2m\omega_{rf}^2}-U_{dc} -U_{bias}\right)}\\
 \omega_z &= \sqrt{\frac{2eU_{dc}}{m}}.
\end{align}

It is convenient to write the potential in terms of the trap frequencies instead of the geometrical dimensions
\begin{align}
\omega_z^2&=\frac{2eU_{dc}}{m}\\
\omega_x^2+\omega_y^2+\omega_z^2&=\frac{e^2U_{rf}^2}{m^2\omega_{rf}^2}\\
\omega_x^2-\omega_y^2&=\frac{2eU_{bias}}{m},
\end{align}
leading to the following replacements of the voltages in terms of the desired effective trap frequencies $(\omega_x, \omega_y, \omega_z)$:
\begin{align}
U_{dc}&=\frac{1}{2}\frac{m}{e}\omega_z^2\\
U_{rf}&=\frac{m}{e}\omega_{rf}\sqrt{\omega_x^2+\omega_y^2+\omega_z^2}\\
U_{bias}&=\frac{1}{2}\frac{m}{e}(\omega_x^2-\omega_y^2).
\end{align}
Therefore, the potential may be written as:
\begin{align}
 \Phi(x,y,z,t) &= \frac{m}{4e} \left[2 \omega_{rf} \sqrt{\omega_x^2 + \omega_y^2 + \omega_z^2} \left( x^2 - y^2 \right) \cos (\omega_{rf} t)
+\omega_z^2 (2z^2 - x^2 - y^2) + \left( \omega_x^2 - \omega_y^2 \right) (x^2 - y^2) \right].
\end{align}
This leads to the following classical equations of motion:
\begin{align}
\ddot{x}&=\left[\frac{1}{2}(-\omega_x^2+\omega_y^2+\omega_z^2)-\omega_{rf}\sqrt{\omega_x^2+\omega_y^2+\omega_z^2}\cos(\omega_{rf}t)\right]x\\
\ddot{y}&=\left[\frac{1}{2}(\omega_x^2-\omega_y^2+\omega_z^2)+\omega_{rf}\sqrt{\omega_x^2+\omega_y^2+\omega_z^2}\cos(\omega_{rf}t)\right]y\\
\ddot{z}&=-\omega_z^2z.
\end{align}

\subsection{Coulomb Repulsion}

The coulomb potential for N ions is given by:

\begin{align}
 \Phi_c &= \displaystyle\sum_{i < j}\frac{e^2}{4 \pi \epsilon_0} \frac{1}{| r_i - r_j |} \\
	&= \displaystyle\sum_{i < j}\frac{e^2}{4 \pi \epsilon_0} \frac{1}{\sqrt{(x_i - x_j)^2 + (y_i - y_j)^2 +  (z_i - z_j)^2}}
\end{align}

This leads to the following accelerations experienced by the particle $i$:
\begin{align}
\ddot{x_i}&=\frac{1}{m} \displaystyle\sum_{j \neq i}\frac{e^2}{4 \pi \epsilon_0} \frac{(x_i - x_j)}{\left((x_i - x_j)^2 + (y_i - y_j)^2 +  (z_i - z_j)^2\right)^{\frac{3}{2}}}\\
\ddot{y_i}&=\frac{1}{m} \displaystyle\sum_{j \neq i}\frac{e^2}{4 \pi \epsilon_0} \frac{(y_i - y_j)}{\left((x_i - x_j)^2 + (y_i - y_j)^2 +  (z_i - z_j)^2\right)^{\frac{3}{2}}}\\
\ddot{z_i}&=\frac{1}{m} \displaystyle\sum_{j \neq i}\frac{e^2}{4 \pi \epsilon_0} \frac{(z_i - z_j)}{\left((x_i - x_j)^2 + (y_i - y_j)^2 +  (z_i - z_j)^2\right)^{\frac{3}{2}}}\\
\end{align}

\subsection{Laser Interaction}

We model the interaction of the laser with a two-level atom using a general form of the Einstein equations (Cohen Tannoudj p286). 
In terms of the populations of the excited and ground states $p_e$ and $p_g$ (called $\sigma_{bb}$ and $\sigma_{aa}$ in the reference):

\begin{align}
\frac{dp_{e}}{dt} &=   -\Gamma p_e + \Gamma_l (p_g - p_e) \\
\frac{dp_{g}}{dt} &=   +\Gamma p_e + \Gamma_l (p_e - p_g)
\end{align}

where $\Gamma$ is the rate of the spontaneous emission and $\Gamma_l$ is the rate of laser interaction, which is proportional to the laser intensity. 
In the steady state, $\frac{dp_{g}}{dt} = \frac{dp_{e}}{dt} = 0$. Using $p_g = 1 - p_e$, we have:

\begin{equation}
p_e= \frac{\Gamma_l}{2 \Gamma_l + \Gamma} = \frac{1}{2} \frac{2 \Gamma_l}{2 \Gamma_l + \Gamma} 
\end{equation}

Our goal is to identify the laser coupling rate $\Gamma_l$ in terms of saturation and detuning. To accomplish this we consider optical bloch equations (see atom photon interaction p369).
The steady state solution is:

\begin{equation}
p_e = \frac{1}{2} \frac{s_{\text{eff}}}{s_{\text{eff}}+1}
\end{equation}

allowing us to identify

\begin{equation}
\Gamma_l = \Gamma \frac{s_{\text{eff}}}{2} 
\end{equation}

where the saturation parameter $s_{\text{eff}}$ is defined in terms of the rabi frequency $\Omega$ and the detuning $\Delta$.

\begin{equation}
s_{\text{eff}} = \frac{\frac{\Omega^2}{2}}{\Delta^2 + \frac{\Gamma^2}{4}}
\end{equation}

It is common to define the saturation parameter $s_0$ as:

\begin{equation}
s_0 = \frac{2 \Omega^2}{\Gamma^2} = s_{\text{eff}} \left(\Delta = 0 \right)
\end{equation}

In this case,

\begin{equation}
s_{\text{eff}} = \frac{s_0}{\left(\frac{2 \Delta}{\Gamma}\right)^2 + 1}
\end{equation}

Then the laser coupling rate is given by:

\begin{equation}
 \Gamma_l = \frac{\Gamma}{2} \frac{s_0}{\left(\frac{2 \Delta}{\Gamma}\right)^2 + 1}
\end{equation}

If we start in the ground state, then per unit time $dt$, the probability for the ion to get excited is:






\end{document}