\documentclass[a4paper,10pt]{article}
\usepackage[utf8]{inputenc}
\usepackage{amsmath}
\usepackage[margin=0.5in]{geometry}

\begin{document}

\section{Theory}

We follow the treatment of Leibfried et al. where the potential of the ion trap is assumed to be harmonic:
\begin{equation}
 \Phi(x,y,z,t) = \frac{1}{2} U_{dc}(2z^2 - x^2 - y^2) + \frac{1}{2}U_{rf}(x^2 - y^2)\cos (\omega_{rf} t)
\end{equation}
In the expression above, $U_{dc}$ is the potential applied to the DC electrodes ($U$ in Leibfried) and $U_{rf}$ refers to the potential on the RF electrodes ($\tilde{U}$ in Leibfried), applied with radial frequency $\omega_{rf}$. We use the special choice of coefficients $\alpha=\beta=-1, \gamma=2$ and $\alpha'=-\beta'=1, \gamma'=0$.

The secular motion of the ion can be described with a pseudopotential approximation. The potential energy yields
\begin{equation}
 \Psi(x,y,z) =  \frac{1}{2}eU_{dc}(2z^2 - x^2 - y^2) + \frac{e^2U_{rf}^2}{4 m \omega_{rf}^2} (x^2 + y^2)
\end{equation}
From this, the trap frequencies are:
\begin{align}
 \omega_x &= \omega_y = \sqrt{\frac{e}{m}\left(\frac{e U_{rf}^2}{2m\omega_{rf}^2}-U_{dc} \right)}\\
 \omega_z &= \sqrt{\frac{2eU_{dc}}{m}}.
\end{align}

In this model, the radial trap frequencies are degenerate, which may lead to poor laser cooling. In order to lift the degeneracy of the radial modes, we
apply an additional static bias $U_{bias}$ on the RF electrodes as follows:
\begin{equation}
 \Phi(x,y,z,t) = \frac{1}{2} U_{dc}(2z^2 - x^2 - y^2) + \frac{1}{2}U_{bias}(x^2 - y^2) + \frac{1}{2}U_{rf}(x^2 - y^2)\cos (\omega_{rf} t)
\end{equation}
The Laplace's equation $\triangle\Phi= 0$ still holds with the additional bias term. The trap frequencies are now non-degenerate and are given by:
\begin{align}
 \omega_x &= \sqrt{\frac{e}{m}\left(\frac{e U_{rf}^2}{2m\omega_{rf}^2}-U_{dc} +U_{bias}\right)}\\
 \omega_y &= \sqrt{\frac{e}{m}\left(\frac{e U_{rf}^2}{2m\omega_{rf}^2}-U_{dc} -U_{bias}\right)}\\
 \omega_z &= \sqrt{\frac{2eU_{dc}}{m}}.
\end{align}

It is convenient to write the potential in terms of the trap frequencies instead of the geometrical dimensions
\begin{align}
\omega_z^2&=\frac{2eU_{dc}}{m}\\
\omega_x^2+\omega_y^2+\omega_z^2&=\frac{e^2U_{rf}^2}{m^2\omega_{rf}^2}\\
\omega_x^2-\omega_y^2&=\frac{2eU_{bias}}{m},
\end{align}
leading to the following replacements of the voltages in terms of the desired effective trap frequencies $(\omega_x, \omega_y, \omega_z)$:
\begin{align}
U_{dc}&=\frac{1}{2}\frac{m}{e}\omega_z^2\\
U_{rf}&=\frac{m}{e}\omega_{rf}\sqrt{\omega_x^2+\omega_y^2+\omega_z^2}\\
U_{bias}&=\frac{1}{2}\frac{m}{e}(\omega_x^2-\omega_y^2).
\end{align}
Therefore, the potential may be written as:
\begin{align}
 \Phi(x,y,z,t) &= \frac{m}{4e} \left[2 \omega_{rf} \sqrt{\omega_x^2 + \omega_y^2 + \omega_z^2} \left( x^2 - y^2 \right) \cos (\omega_{rf} t)
+\omega_z^2 (2z^2 - x^2 - y^2) + \left( \omega_x^2 - \omega_y^2 \right) (x^2 - y^2) \right].
\end{align}
This leads to the following classical equations of motion:
\begin{align}
\ddot{x}&=\left[\frac{1}{2}(-\omega_x^2+\omega_y^2+\omega_z^2)-\omega_{rf}\sqrt{\omega_x^2+\omega_y^2+\omega_z^2}\cos(\omega_{rf}t)\right]x\\
\ddot{y}&=\left[\frac{1}{2}(\omega_x^2-\omega_y^2+\omega_z^2)+\omega_{rf}\sqrt{\omega_x^2+\omega_y^2+\omega_z^2}\cos(\omega_{rf}t)\right]y\\
\ddot{z}&=-\omega_z^2z.
\end{align}

\end{document}