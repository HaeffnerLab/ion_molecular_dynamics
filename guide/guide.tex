\documentclass[a4paper,10pt]{article}
\usepackage[utf8]{inputenc}
\usepackage{amsmath}
\usepackage[margin=0.5in]{geometry}

\begin{document}

\section{Theory}

We follow the treatment of Zhang (PRA 76 012719) where the potential of the ion trap is assumed to be harmonic:

\begin{equation}
 V(x,y,z,t) = \frac{2V_{rf}}{{r_0}^2}(x^2 - y^2)\cos (\Omega t) + \frac{\kappa V_{dc}}{2 {z_0}^2}(2z^2 - x^2 - y^2)
\end{equation}

In the expression above, $V_{rf}$ refers to the voltage on the RF electrodes ($V_0$ in Zhang), applied with radial frequency $\Omega$. Dimensions $r_0$ and $z_0$ are radial and axial dimensions of the trap
and $\kappa$ is a constant determined by trap geometry. Finally, $V_{dc}$ is the voltage applied to the DC electrodes ($V_{ec}$ in Zhang).

The secular motion of the ion can be described with a pseudopotential approximation:

\begin{equation}
 \Psi(x,y,z) = \frac{e {V_{rf}}^2}{ m \Omega^2 {r_0}^4} (x^2 + y^2) + \frac{\kappa V_{dc}}{2 {z_0}^2}(2z^2 - x^2 - y^2)
\end{equation}

From this, the trap frequencies are:

\begin{align}
 \omega_x &= \omega_y = \sqrt{\frac{Q}{m}\left(\frac{q V_{rf}}{4 {r_0}^2} - \frac{\kappa V_{dc}}{{z_0}^2} \right)}\\
 \omega_z &= \sqrt{\frac{Q}{m}\frac{2 \kappa V_{dc}}{ {z_0}^2 }}
\end{align}

where the stability parameter $q$ is defined as:

\begin{equation}
 q = \frac{2 Q V_{rf}}{ m r^2_0 \Omega^2}
\end{equation}

In this model, the radial trap frequencies are degenerate, which may lead to poor laser cooling. In order to lift the degeneracy of the radial modes, we
apply an additional static bias $V_{bias}$ on the RF electrodes as follows:

\begin{equation}
 V(x,y,z,t) = \frac{2V_{rf}}{{r_0}^2}(x^2 - y^2)\cos (\Omega t) + \frac{\kappa V_{dc}}{2 {z_0}^2}(2z^2 - x^2 - y^2) + \frac{\kappa V_{bias}}{2 {z_0}^2}(x^2 - y^2)
\end{equation}

The Laplace's equation $\triangle V = 0$ still holds with the additional bias term. The trap frequencies are now non-degenerate and are given by:

\begin{align}
 \omega_x &= \sqrt{\frac{Q}{m}\left(\frac{q V_{rf}}{4 {r_0}^2} - \frac{\kappa (V_{dc} - V_{bias})}{{z_0}^2} \right)}\\
 \omega_y &= \sqrt{\frac{Q}{m}\left(\frac{q V_{rf}}{4 {r_0}^2} - \frac{\kappa (V_{dc} + V_{bias})}{{z_0}^2} \right)}\\
 \omega_z &= \sqrt{\frac{Q}{m}\frac{2 \kappa V_{dc}}{ {z_0}^2 }}
\end{align}

It is convinient to write the potential in terms of the trap frequencies instead of the geometrical dimensions:

\begin{align}
 \frac{\kappa V_{dc}}{ 2 {z_0}^2} &= \frac{1}{4} \left( \frac{m}{Q} \right) {\omega_z}^2 \\
 \frac{\kappa V_{bias}}{ 2 {z_0}^2} &= \frac{1}{4} \left( \frac{m}{Q} \right) {\omega_{bias}}^2 \\
 \frac{2V_{rf}}{{r_0}^2} &= \sqrt{8} \Omega \left( \frac{m}{Q} \right) \omega_\bot \\
 \omega_x &= \sqrt{\omega_\bot^2 - \frac{1}{2} \omega_z^2 + \frac{1}{2} \omega_{bias}^2} \\
 \omega_y &= \sqrt{\omega_\bot^2 - \frac{1}{2} \omega_z^2 - \frac{1}{2} \omega_{bias}^2}
\end{align}

If one is given $(\omega_x, \omega_y, \omega_z)$, then:

\begin{align}
  \omega_{bias}^2 = \omega_x^2 - \omega_y^2\\
  \omega_\bot = \sqrt{\omega_x^2 + \omega_y^2 + \omega_z^2}
\end{align}

Therefore, the potential may be written as:

\begin{align}
 V(x,y,z,t) &= \sqrt{8} \Omega \left( \frac{m}{Q} \right)\sqrt{\omega_x^2 + \omega_y^2 + \omega_z^2} \left( x^2 - y^2 \right) \cos (\Omega t)  
 + \frac{1}{4} \left( \frac{m}{Q} \right) {\omega_z}^2(2z^2 - x^2 - y^2)
 + \frac{1}{4} \left( \frac{m}{Q} \right) \left( \omega_x^2 - \omega_y^2 \right) (x^2 - y^2) \\
	    &= \frac{1}{4} \left( \frac{m}{Q} \right) \left[\sqrt{128} \Omega \sqrt{\omega_x^2 + \omega_y^2 + \omega_z^2} \left( x^2 - y^2 \right) \cos (\Omega t)
	    +\omega_z^2 (2z^2 - x^2 - y^2) + \left( \omega_x^2 - \omega_y^2 \right) (x^2 - y^2)
	    \right]
\end{align}

\end{document}
